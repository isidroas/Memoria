\chapter{Simulador del estimador de estados}\label{chp:simu}

El siguiente archivo también puede verse y descargarse en el repositorio \url{https://github.com/isidroas/quadrotor\_simulator}

%\begin{codigo}{Simulador del estimador de estados  (\textit{main.py})}
\inputminted[bgcolor=black!5!white]{python}{apendices/quadrotor_simulator/main.py}
%\end{codigo} 

\chapter{Detector de marcadores visuales}

Los siguientes archivos se encuentran también en \url{https://github.com/isidroas/rpi\_vision\_uav}

\section{main.cpp}
\inputminted[bgcolor=black!5!white]{c++}{apendices/rpi_vision_uav/main.cpp}

\section{vision\_params.yml}\label{sec:vision-params}
\inputminted[bgcolor=black!5!white]{yaml}{apendices/rpi_vision_uav/vision_params.yml}

\section{marker\_vision.h}

\newcounter{lineInvertPoseA}
\newcounter{lineInvertPoseB}
\setcounter{lineInvertPoseB}{291}
\setcounter{lineInvertPoseA}{\thelineInvertPoseB -7}

\inputminted[bgcolor=black!5!white,lastline=\thelineInvertPoseA]{c++}{apendices/rpi_vision_uav/marker_vision.h}

La siguiente función invierte la posición y la rotación. También corrige la posición de la cámara con respecto al UAV. Sus argumentos son:
\begin{itemize}
\item rvec.  Vector de entrada. Vector de rotación del marcador con respecto a los ejes de la cámara
\item tvec.  Vector de entrada. Translación del marcador con respecto a los ejes de la cámara
\item pos.   Vector de salida. Posición del uav/cámara con respecto al marcador
\item eul.   Vector de salida. Orientación del uav con respecto al marcador. El orden de los elementos son 0: roll, 1: pitch, 2: yaw
\end{itemize}

\inputminted[bgcolor=black!5!white,firstline=\thelineInvertPoseB]{c++}{apendices/rpi_vision_uav/marker_vision.h}

\section{mavlink\_helper.h}
\inputminted[bgcolor=black!5!white]{c++}{apendices/rpi_vision_uav/mavlink_helper.h}
