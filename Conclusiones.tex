\chapter{Conclusiones} \label{chp-conclusiones}
%\lettrine[lraise=-0.1, lines=2, loversize=0.2]{S}{e} ha podido ver que el \textbf{autopiloto} \textit{PX4} ofrece una gran cantidad de funcionalidades, que vienen validadas por la multitud de vuelos que se realizan con este autopiloto. 

\chapter{Trabajos futuros}
%\lettrine[lraise=-0.1, lines=2, loversize=0.2]{L}{os} trabajos futuros vendrán marcados por la importante experiencia que me ha supuesto el desarrollo de este proyecto, tanto en la investigación y estudio de trabajos ya realizados, como en mi experimentación de prueba y error que me han ayudado en mi objetivo de llevar a buen puerto este Trabajo de Fin de Grado.
%A continuación, relaciono una serie de elementos y acciones  en los que me gustaría profundizar en futuros proyectos:

\begin{itemize}
\item Disminuir la necesidad de colocar marcadores cada poca distancia. Se podría usar lentes ojo de pez, varias cámaras (abajo y delante) o un gimbal.  
\item Es verdad que colocar muchos marcadores puede ser no ser viable, pero por mucho que avanze la robótica siempre será mejor los marcadores artificiales a los naturales, para aquellas aplicaciones que se requieran una precisión extra.
\end{itemize}

\endinput
