% !TEX root =../LibroTipoETSI.tex
\chapter*{Resumen}
\pagestyle{especial}
\chaptermark{Resumen}
\phantomsection
\addcontentsline{toc}{listasf}{Resumen}

\lettrine[lraise=-0.1, lines=2, loversize=0.2]{E}{n} este proyecto  se va a desarrollar el  estudio de  un vehículo aéreo no tripulado que tiene dos hélices o rotores orientables, denominado tiltrotor, del ingles \textit{tilt} que significa inclinar.

El fundamento de este trabajo es contribuir  en el estudio de una nave convertible en la  que los rotores, al ser  inclinables, funcionan con las ventajas de   dos modelos de aeronaves diferentes, una del tipo helicóptero que lo dota de alta maniobrabilidad,  y otra de tipo aeroplano que le permite recorrer largas distancias.

Entendiendo la importancia de la simulación en la construcción de vehículos aéreos, se  opta por la elección de dos herramientas diferentes en el ámbito de la simulación. Se trata de comprobar y testar las coincidencias entre ellas, de manera que se corrija y  disminuya la posibilidad de errores  en el proceso.

Por último, se  añade al proyecto la fabricación de un tiltrotor para verificar los modelos utilizados en la simulación. 



\chapter*{Abstract}
\pagestyle{especial}
\chaptermark{Abstract}
\phantomsection
\addcontentsline{toc}{listasf}{Abstract}

\lettrine[lraise=-0.1, lines=2, loversize=0.2]{T}{he} basis of this work is to contribute to the study of a convertible aerial vehicle in which the rotors, being tiltable, operate with the advantages of two different aircraft models, one of the helicopter type that gives it high maneuverability, and one of the airplane type that allows traveling long distances.

Understanding the importance of simulation in the design of aerial vehicles, we have chosen two different tools in the field of simulation. It is about verifying and testing the coincidences between them, so that it corrects and reduces the possibility of errors in the process.

Finally, a prototype will be built in order to verify the models used in the simulation.
