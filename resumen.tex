% !TEX root =../LibroTipoETSI.tex
\chapter*{Resumen}
\pagestyle{especial}
\chaptermark{Resumen}
\phantomsection
\addcontentsline{toc}{listasf}{Resumen}


\lettrine[lraise=-0.1, lines=2, loversize=0.2]{E}{n} este trabajo se busca conseguir un posicionamiento preciso de un quadrotor intentando no usar componentes demasiado costosos. Para ello, se han usado unos marcadores impresos y se ha dotado al vehículo de una cámara. Del proceso seguido hasta conseguirlo, se detallará tanto su programación como los componentes físicos elegidos.

Por otra parte, se analiza una de las mejoras a un estimador de estados convencional que se le realizan en la práctica, tomando como ejemplo el autopiloto de código abierto \textit{PX4}. Una mejora que resuelve la duda de cómo afrontar el retraso que tienen algunos sensores, que se considera nulo en muchos estudios, pero que pueden impactar negativamente en el desempeño si no se trantan adecuadamente. Para demostrar esta idea, se elabora una simulación que pone de manifiesto su utilidad. 

 


\chapter*{Abstract}
\pagestyle{especial}
\chaptermark{Abstract}
\phantomsection
\addcontentsline{toc}{listasf}{Abstract}

\lettrine[lraise=-0.1, lines=2, loversize=0.2]{T}{he} aim of this work is to achieve the precise positioning of a quadrotor attempting not to use expensive components. To that end, printed markers has been used and the vehicle has been provided with a camera. Both programming and the chosen physical components will be detailed.

Moreover, it is analyzed an improvement over standard state estimators, that is carried out in practice, specifically in the open source autopilot \textit{PX4}. This enhancement handle the delay that some sensors have, which is not considered in many studies, but they could have a negative impact in the performance if they are not faced appropriately. This idea is tested in a simulation that illustrate his usefulness.

