% !TEX root =../LibroTipoETSI.tex
\chapter*{Resumen}
\pagestyle{especial}
\chaptermark{Resumen}
\phantomsection
\addcontentsline{toc}{listasf}{Resumen}


%- Se detalla tanto su programación como los componentes elegidos.

\lettrine[lraise=-0.1, lines=2, loversize=0.2]{E}{n} este trabajo se busca conseguir un posicionamiento preciso de un quadrotor intentando no usar componentes demasiado costosos. Para ello, se han usado unos marcadores impresos y se ha dotado al vehículo de una cámara. De este se detallará tanto de su programación como de los componentes físicos elegidos.

Por otra parte, se analiza una de las mejoras que se le realizan en la práctica, a un estimador de estados convencional. En concreto, en cómo afrontar el retraso que tienen algunos sensores, que se considera nulo en muchos estudios, pero que pueden impactar negativamente en el desempeño si no se trantan adecuadamente. Para demostrar esta idea, se elabora una simulación que pone de manifiesto su utilidad. 



\chapter*{Abstract}
\pagestyle{especial}
\chaptermark{Abstract}
\phantomsection
\addcontentsline{toc}{listasf}{Abstract}

%\lettrine[lraise=-0.1, lines=2, loversize=0.2]{T}{he} basis of this work is to contribute to the study of a convertible aerial vehicle in which the rotors, being tiltable, operate with the advantages of two different aircraft models, one of the helicopter type that gives it high maneuverability, and one of the airplane type that allows traveling long distances.

This work seeks to achieve a precise positioning of a quadrotor trying not to use too expensive components. For this, printed markers have been used and the vehicle has been equipped with a camera. This will detail both its programming and the chosen physical components.

On the other hand, one of the improvements made in practice to a conventional state estimator is analyzed. Specifically, in how to deal with the delay that some sensors have, which is considered null in many studies, but which can negatively impact performance if not treated properly. To demonstrate this idea, a simulation is developed that shows its usefulness.

{\color{red} [Pendiente de revisión]}

