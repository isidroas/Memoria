% !TEX root =../main.tex
\chapter{Estimador PX4}\label{chp-01}

\lettrine[lraise=-0.1, lines=2, loversize=0.2]{E}{n} muchas ocasiones se tienen sensores con un retraso y una frecuencia de actualización muy diferentes entre ellos, por ejemplo una IMU es mucho más rápida que el procesamiento de la imagen de una cámara. PX4 lo soluciona añadiendo más elementos a la estructura original de un estimador de estados. Uno de los elementos es un \textit{Filtro de Kalman Extendido} (EKF). Este no usa las medidas más nuevas que le llegan, si no que las almacena y utiliza las que llegaron hace un determinado tiempo. Corriendo en paralelo pero a una frecuencia mayor, existe un estimador llamado \textit{Filtro de Salida}, el cual sí que utiliza la última medida del acelerómetro y del giróscopo. 

Supongamos que se tiene un sistema que se mueve en el espacio del que se quiere conocer sus estados, en concreto, su posición, su velocidad y su ángulo. Para este objetivo se disponen de sensores que son: acelerómetro, giróscopo, GNSS y posición por visión. Cada uno de ellos tiene diferentes propiedades en cuanto a retraso, ruido, etc. que se muestran en la siguiente tabla: 

\begin{tabular}{|c|c|c|}
Sensor			& Retraso (ms) 	& Ruido 	\\ \hline 
acelerómetro 		& 0 		& 	      	\\ 
giróscopo 		& 0 		& 	      	\\ 
GNSS      		& 		& 	      	\\ % eliminar? 
Posición por visión 	& 30 		& 0.03 	
% Flujo óptico?
\end{tabular}

Se puede notar que la posición por visión es una fuente muy precisa de posición, pero sin embargo tiene un gran retraso desde que se toma la imagen hasta que se procesa, como suele suceder en la práctica. 

En el primera ejecución del filtro, después de inicilizar los estados, se toma una medida de la IMU (acelerómetro y giróscopo). Esta la utiliza el \textit{Filtro de Salida} y en el EKF todavía no, si no que se guarda en su buffer. 
% Mostrar figura con la primera medida.

Se ejecutan 6 periodos y por primera vez se utilizan las medidas de la IMU en el EKF. Los estados que se generan se utilizan para corregir al Filtro de Salida (ahora mismo no tiene mucho sentido ya que los dos disponen de los misma información). Los estados de este último filtro se han estado guardando también en un buffer, del que se coge la medida más antigua y se calcula su diferencia con los estados del EKF. Esta cantidad se multiplica por una ganancia se le suma a todos los elementos del buffer de salida.  

Se ejecuta 4 periodos más, que equivalen a 20 ms y llega una medida de posición visual, pero esta no se coloca junto con las medidas más recientes de la IMU, si no que se retrasa de acuerdo al parámetro de la tabla, que en este ejemplo son 30 ms.  

\section{EKF}
Estados:
\begin{align}
X = 
\begin{bmatrix} 
x \\ y \\ V_x \\ V_y \\ \theta
\end{bmatrix}
\end{align}

Modelo de predicción (modelo cinemático, no dinámico):
\begin{align}
\begin{bmatrix} 
x \\ y 
\end{bmatrix}_{k+1}
=
\begin{bmatrix} 
x \\ y 
\end{bmatrix}_k
+
\begin{bmatrix} 
V_x \\ V_y 
\end{bmatrix}_k
\Delta t
\end{align}

\begin{align}
\begin{bmatrix} 
V_x \\ V_y 
\end{bmatrix}_{k+1}
=
\begin{bmatrix} 
V_x \\ V_y 
\end{bmatrix}_k + 
\Delta t
\begin{bmatrix} 
\cos{\theta} & \sin{\theta} \\ -\sin{\theta} & \cos{\theta}
\end{bmatrix}
\bm{a} +  
\begin{bmatrix} 
0 \\ - m\ g 
\end{bmatrix}\Delta t
\end{align}

\begin{align}
\theta_{k+1} = \theta_k + \Delta t \omega
\end{align}


Jacobiano del modelo de predicción:
\begin{align}
F = 
\begin{bmatrix} 
%x/X
1 	&0	&\Delta t	&0		&0\\
%y/X
0 	&1	&0		&\Delta t	&0\\
%Vx/X
0 	&0	&1		&0		&\Delta t\left(-a_x\sin{\theta} + a_y\cos{\theta}\right) \\
%Vy/X
0 	&0	&0		&1		&\Delta t\left(-a_x\cos{\theta} - a_y\sin{\theta}\right) \\
%theta/X
0 	&0	&0		&0		&1
\end{bmatrix}
\end{align}

Jacobiano del acelerómetro y el giróscopo
\begin{align}
G = 
\begin{bmatrix} 
%x/a,w
0 			&0			&0\\
%y/a,w
0 			&0			&0\\
%Vx/a,w
\Delta t \cos{\theta} 	&\Delta t \sin{\theta}	&0\\
%Vy/a,w
-\Delta t \sin{\theta} 	&\Delta t \cos{\theta}	&0\\
%theta/a,w
0 			&0			&1		
\end{bmatrix}
\end{align}

Matriz de covarianzas de la predicción:
\begin{align}
Q = 
G
\begin{bmatrix} 
\sigma^2_a 	& 0 		& 0\\
0 		& \sigma^2_a 	& 0\\
0 		& 0 		& \sigma^2_\omega\\
\end{bmatrix}
G^T
\end{align}


\endinput
