% !TEX root =../LibroTipoETSI.tex
%El anterior comando permite compilar este documento llamando al documento raíz
\chapter{Introducción}\label{chp-02}

\lettrine[lraise=-0.1, lines=2, loversize=0.2]{E}{l} interés 

En la robótica uno de los mayores problemas es la percepción del entorno. Más en concreto en la aérea son la duración de las baterias y su recambio que no es automático

Posibles aplicaciones del posicionamiento con precisión centimétrica:
- Despegue y aterrizaje
- Trayectorias cerca de obstáculo

Esto junto con una mejora de la duración de las baterías o de su cambio automático, podría llevar a
- Recogida y depósito de paquetes de manera autónoma
- Vuelta de reconocimiento para aplicaciones de seguridad cuando se detecte un posible intruso.

En este trabajo se propone conseguir ese posicionamiento mediante marcadores visuales. No cómo única fuente de posición, sino combinándola con otras como el GPS.

Hay alternativas pero  son más caras:
- Balizas acústicas
- GNSS RTK


% Trabajos anteriores
La empresa \textit{Everdrone} es una de las que más a avanzado en este campo. 
% Flytbase también está relacionado


\endinput
