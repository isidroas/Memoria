% !TEX root =../LibroTipoETSI.tex
%El anterior comando permite compilar este documento llamando al documento raíz
\chapter{Introducción}\label{chp-02}

\ifthenelse{\entregar = 1}{
{\color{red} [Pendiente] }
}{
\lettrine[lraise=-0.1, lines=2, loversize=0.2]{E}{n} la robótica, uno de los mayores problemas es la percepción del entorno y su ubicación en él. En concreto, cuando se habla de vehículos aéreos no tripulados (UAV), suelen poder posicionarse en el exterior con una precisión de metros. Sin embargo, cuando se trata de navegar en interiores o con precisión más alta el coste de los componentes puede ser muy elevados. Por ejemplo, en exteriores, se puede utilizar la \textit{navegación cinética satelital en tiempo real} (RTK) o en interiores se pueden utilizar balizas acústicas.

%Existen formas pero son más caras:
%- Balizas acústicas
%- GNSS RTK

% Marcadores visuales. Solución robusta (no hay muchos falsos positivos) ya que es complicado encontrarse en el entorno algo parecido (no como los que usan una pelota), y además barata ya que tiene poco costo computacional, al contrario que los SLAM por ejemplo. 

Posibles aplicaciones del posicionamiento con precisión centimétrica:
- Despegue y aterrizaje
- Trayectorias cerca de obstáculo

Esto junto con una mejora de la duración de las baterías o de su cambio automático, podría llevar a
- Recogida y depósito de paquetes de manera autónoma
- Vuelta de reconocimiento para aplicaciones de seguridad cuando se detecte un posible intruso.


En este trabajo se propone conseguir ese posicionamiento mediante marcadores visuales. No cómo única fuente de posición, sino combinándola con otras como el GPS.



% Trabajos anteriores
La empresa \textit{Everdrone} es una de las que más a avanzado en este campo. 
% Flytbase también está relacionado

% TODO: Los puntos que se veran
En este trabajo en primer lugar, en el capítulo \ref{ch-1} se hace un estudio de una parte clave para el posicionamiento con marcadores, que es el estimador de estados.  
% además de simulaciones
En el capítulo \ref{ch-2} se explica cómo se ha implementado este posicionamiento, mostrando los componentes utilizados y explicando el código escrito.
}


\endinput
