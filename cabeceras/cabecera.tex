%Minion=true, English=true, Myfinal=true

%:Paquete de estilos propuesto
\usepackage{cabeceras/libroETSI}

%:Paquete específico para cargar tikz (y sus librerías) y pgfplots
\usepackage{cabeceras/dtsc-creafig}

%:Paquete para notaciones específicas
\usepackage{cabeceras/notacion}

%:Paquete para incorporar aspectos concretos de la edición
\usepackage{cabeceras/edicionPFC}


%:Estas líneas de código son INNECESARIAS excepto para mostrar determinadas características en este manual. Pueden eliminarse o comentarse sin ningún problema.
%Se usan para compilar el capítulo estilolibroetsi.tex
\usepackage[final]{showexpl}
\lstset{explpreset={frame=none,rframe={}, numbers=none,numbersep=3pt, columns=flexible,language={[LaTeX]TeX},basicstyle=\ttfamily,keywordstyle=\color{blue}}}%numberstyle=\tiny,

%:Para modificar fácilmente la fuente del texto.
\makeatletter
\ifdtsc@Minion % Queremos utilizar la fuente Minion y lo hemos declarado al principio
	\ifluatex
		\setmainfont[Renderer=Basic, Ligatures=TeX,	% Fuente del texto 
		Scale=1.01,
		]{Minion Pro}
   		% En este caso conviene modificar ligeramente el tamaño de las fuentes matemáticas
		\DeclareMathSizes{10}{10.5}{7.35}{5.25}
		\DeclareMathSizes{10.95}{11.55}{8.08}{5.77}
		\DeclareMathSizes{12}{12.6}{8.82}{6.3}
%		\setmainfont[Renderer=Basic, Ligatures=TeX,	% Fuente del texto 
%		]{Adobe Garamond Pro}
%		\setmainfont[Renderer=Basic, Ligatures=TeX,	% Fuente del texto 
%		]{Palatino LT Std}
	\fi
\else
	\ifluatex
		% Para utilizar la fuente Times New Roman, o alguna otra que se tenga instalada
		\setmainfont[Renderer=Basic, Ligatures=TeX,	% Fuente del texto 
		Scale=1.0,
		]{Times New Roman}
	\else
		\usepackage{tgtermes} 	%clone of Times
		%\usepackage[default]{droidserif}
		%\usepackage{anttor} 	
	\fi
\fi
\makeatother

\usepackage[
    backend=biber,
    sorting=none, 
    %style=authoryear-icomp,
    %sortlocale=de_DE,
    %natbib=true,
    %url=false, 
    %doi=true,
    %eprint=false
]{biblatex}
\defbibheading{etsi}[]{%
        \chapter*{Bibliografía}%
        \chaptermark{Bibliografía} 
        \markboth{#1}{#1}}
\addbibresource{bibliografia.bib}

% Ejemplo de Glosario
\newacronym[type=main]{ETSI}{ETSI}{Escuela Técnica Superior de Ingeniería}
\newacronym[type=main]{US}{US}{Universidad de Sevilla}
\newacronym[type=main]{DMC}{DMC}{Canal Discreto sin Memoria}


\makeindex
\makeglossaries %Si no se quiere el glosario, comentar esta línea.

% Formato A4
\geometry
{paperheight=297mm,%
paperwidth=210mm,%
top=25mm,%
headsep=8.5mm,%
includefoot, 
textheight=240mm, 
textwidth=150mm, 
bindingoffset=0mm, 
twoside}

\usepackage[a4,center]{crop}%para poner las cruces de esquina de página, poner la opción cross

%:Esquema de numeración por defecto
\setenumerate[1]{label=\normalfont\bfseries{\arabic*.}, leftmargin=*, labelindent=\parindent}
\setenumerate[2]{label=\normalfont\bfseries{\alph*}), leftmargin=*}
\setenumerate[3]{label=\normalfont\bfseries{\roman*.}, leftmargin=*}
\setlist{itemsep=.1em}
\setlength{\parindent}{1.0 em}

\setcounter{tocdepth}{4}						% El nivel hasta el que se muestra el índice 

\usepackage{morewrites}
\usepackage[outputdir=build]{minted}
\setminted{fontsize=\footnotesize,linenos,breaklines}
\usepackage[minted]{tcolorbox}
\tcbuselibrary{breakable,skins}

\newtcolorbox[auto counter,number within=section]{codigo}[2][]{enhanced,breakable, left=1cm,colback=black!5!white,colframe=blue!0!black,fonttitle=\bfseries, title=Código ~\thetcbcounter: #2,#1}
\usepackage{adjustbox}
\usepackage{tikz-3dplot}
%\usepackage{subcaption}
\setcounter{secnumdepth}{40}

\usepackage{gensymb}

\usepackage{pdflscape}
\usepackage{multirow}
\usepackage{placeins}

\makeatletter
\providecommand\add@text{}
\newcommand\tagaddtext[1]{%
     \gdef\add@text{#1\gdef\add@text{}}}% 
     \renewcommand\tagform@[1]{%
          \maketag@@@{\llap{\add@text\quad}(\ignorespaces#1\unskip\@@italiccorr)}%
     }
\makeatother
